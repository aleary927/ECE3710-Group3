\documentclass{subfile} 

\begin{document}
  The overall setup of the application is by considering the music as broken down into 
  windows of a half second each. 
  For each half second window, each drum pad input is either expected to be struck, or not 
  struck.
  The millisecond counter is used to determine when the next window has been reached, which 
  in turn causes score to be updated, and then the window information to be updated to reflect 
  the next window.

  \subsection{Software Layout} 
  The procedures are organized into categories, to keep good separation of concerns. 
  There are procedures for processing the drum pad inputs, keeping track of current window and 
  other synchronization information, keeping track of score, and updating the display. 
  These procedures all update global data structures, which are a simple way to share information. 
  This setup means that the synchronization procedures are the only ones accessing the millisecond 
  counter, and the drum pad processing procedures are the only ones accessing the drum pad I/O address.

  \subsection{Synchronization} 

  \subsection{Score Keeping}

  \subsection{Visual Update}
\end{document}
