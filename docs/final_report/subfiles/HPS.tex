\documentclass{subfile} 

\begin{document}
  The ARM cores which compose the HPS (hard processor system) of the Cyclone V SoC (system on chip) FPGA 
  are tightly integrated with the FPGA with over 1000 interconnecting signals. % TODO reference needed
  The most notable of these connections are three bridges, allowing for data to be transferred 
  easily and efficiently.

  The reason for utilizing the HPS in this design was to keep the external hardware requirements 
  to nothing beyond the drum pads. 
  An FPGA is not a device meant to store large amounts of data at once, meaning that large audio 
  files could not be hosted on the FPGA's limited onboard block memory.

  The bridge utilized for this design is known as the "hps2fpgalw bridge". 
  
  \subsection{Instantiating HPS Connections}
  The HPS must be initiated using megafunctions built into Quartus. 
  These megafunctions can be instantiated by using the platform designer to generate IP.
  Even with the help of the platform designer, this can be very tedious as it requires a lot of 
  information about the hardware setup of the exact board which is being used. (including things like memory timing parameters) 
  The best solution is to use a reference design which already has these complicated parameters 
  preconfigured, luckily the board designers include a HPS reference design in the CD ROM. 
  This design is known as the "golden hardware reference design" and includes the HPS connections 
  already instantiated, with the proper configuration for the DE1-SoC board.
  This reference design includes a top-level module for the 

  \subsection{Transfer Protocol} 
  In the interest of keeping this interface as simple as possible, the protocol for transferring 
  data from the hps is simply done by flipping bits on either side.
  When the audio mixer requests a sample from the HPS, it flips a bit to indicate the request.
  Similarly, once the HPS gets the sample and writes it out to the memory location which corresponds to 
  the bridge, the HPS also flips a bit on the end of the sample data, indicating that a new sample 
  is ready. 
  The mixer can read this bit, and determine that the HPS sample has been updated, allowing it to continue and 
  write the completed sample out to the codec.

  One of the connections between the FPGA and the HPS is an interrupt interconnect, allowing for custom interrupts 
  to be received by the HPS, and handled by custom interrupt handlers. 
  An interrupt handler is inserted into the kernel as a kernel module, allowing for modification of interrupt handlers 
  while the HPS is booted up. 
  This solution is technically better because it does not involve a busy loop, and interrupts should not be effected by 
  what processes are running on the system, but this solution would have added unnecessary complexity.
\end{document}
