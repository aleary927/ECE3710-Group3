\documentclass{subfile} 

\begin{document}
  The ARM cores which compose the HPS of the Cyclone V SoC aboard the 
  DE1-Soc
  are tightly integrated with the FPGA with over 1000 interconnecting signals.\cite{cv_hps} 
  The most notable of these connections are three bridges, allowing for data to be transferred 
  easily and efficiently.

  The reason for utilizing the HPS in this design was to keep the external hardware requirements 
  to nothing beyond the drum pads. 
  Since an FPGA is not a device meant to store large amounts of data at once, the soundtrack that 
  is needed to implement the game would not be able to fit on the 
  FPGA's limited onboard block memory, leading to the need to store the data externally.

  \subsection{Instantiating HPS Connections}
  The HPS must be initiated using megafunctions built into Quartus. 
  These megafunctions can be instantiated by using the platform designer to generate IP.
  Even with the help of the platform designer, this can be very tedious as it requires a lot of 
  information about the hardware setup of the exact board which is being used (including things like memory timing parameters).
  The best solution to this issue is to use a reference design which already has 
  these complicated parameters 
  preconfigured. 
  Luckily, the board designers include a HPS reference design in the CD ROM. 
  This design is known as the "golden hardware reference design" and includes the HPS connections 
  already instantiated, with the proper configuration for the DE1-SoC board.

  Connections between the FPGA and HPS can take many forms, but for this project the connections 
  are all in parallel on the FPGA side. 
  On the HPS side, connections are simply memory-mapped. 
  The specific addresses used for connections are generated by a script which is included in 
  the Quartus installation.

  Once the reference design is acquired, custom modules may simply be instantiated within the 
  top-level module of the reference design. 
  In the interests of keeping things organized, it makes the most sense to keep a top-level 
  module for custom hardware, so that only one module is instantiated in the reference 
  design's top-level module.

  \subsection{Transfer Protocol} 
  In the interest of keeping this interface as simple as possible, the protocol for transferring 
  data from the HPS is simply done by flipping bits on either side.
  When the audio mixer requests a sample from the HPS, it flips a bit to indicate the request.
  Similarly, once the HPS produces the sample and writes it out to the bridge,
  it also flips a bit, indicating that the new sample is ready. 
  The mixer can read this bit, and determine that the HPS sample has been updated, 
  allowing it to continue and write the completed sample out to the codec.

  One of the connections between the FPGA and the HPS is an interrupt interconnect, 
  allowing for custom interrupts to be received by the HPS, and handled by custom interrupt handlers. 
  An interrupt handler is inserted into the kernel as a kernel module, allowing 
  for modification of interrupt handlers while the HPS is booted up. 
  This solution is technically better because it does not involve a busy loop, 
  and interrupts should not be effected by what processes are running on the system. 
  Also, interrupts would likely have better latency for producing a sample,
  but this solution would have added unnecessary complexity to the design.
\end{document}
