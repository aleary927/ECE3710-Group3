\documentclass{article}
\usepackage[utf8]{inputenc}
\usepackage{subfiles}

\begin{document}

Rhythm games are a popular genre of interactive entertainment, challenging players to synchronize their inputs with visual and auditory cues. This project, titled \textit{Music Tile}, implements a rhythm game on the FPGA DE10 board, integrating both hardware and software components to create an engaging and interactive experience.

\subsection{Background}

Rhythm games, such as \textit{Guitar Hero} and \textit{Dance Dance Revolution}, have gained massive popularity due to their engaging gameplay and reliance on precise timing. However, implementing such a game using FPGA systems offers unique challenges, including real-time signal processing and hardware-based input handling, which allow for low-latency, highly responsive gameplay.

\subsection{Project Goals}

The primary goal of this project was to implement a fully functional rhythm game on the DE10 board. The game includes:
\begin{itemize}
    \item A VGA-based visual interface for displaying falling notes.
    \item Input handling using four physical drum pads.
    \item Accurate timing synchronization between notes and user inputs.
\end{itemize}

\subsection{Significance}

By utilizing FPGA technology, this project leverages the low-latency, real-time processing capabilities that are essential for rhythm games. The ability to design custom hardware for input handling and timing synchronization ensures a highly responsive gameplay experience, a key feature that cannot be easily achieved with general-purpose processors.

\end{document}
