\documentclass{article}
\usepackage[utf8]{inputenc}
\usepackage{subfiles}

\begin{document}
Rhythm games are a popular genre of interactive entertainment, challenging players to
maintain a consistent rhythm by providing them with visual and auditory queues to 
interact with. 
This project, titled \textit{Music Tiles}, implements a rhythm game on the Terasic DE1-SoC board,
integrating both hardware and software components to create an engaging and interactive experience.
More specifically, a custom-built CPU, using the ECE 3710 architecture (based on CR16) is 
implemented in the FPGA fabric and is used as the controller for this game.
Additionally, several other supporting hardware components such as audio mixing and 
playback, and a VGA graphics controller are designed and used to bring the game to life. 

\subsection{Background}

Rhythm games, such as \textit{Guitar Hero} and \textit{Dance Dance Revolution}, have 
gained massive popularity due to their engaging gameplay and reliance on precise timing. 
However, implementing such a game using FPGA systems offers unique challenges, 
including real-time signal processing and input handling 
of external hardware devices, which allow for low-latency, highly responsive gameplay.

\subsection{Project Goals}
The primary goal of this project was to implement a fully functional rhythm game on an FPGA 
platform. The game includes:
\begin{itemize}
    \item A VGA-based visual interface for displaying notes visually.
    \item Input handling using four physical drum pads.
    \item Audio mixing and playback.
    \item Accurate timing synchronization between notes, user inputs, and a backing audio soundtrack.
\end{itemize}

\subsection{Significance}

By utilizing FPGA technology, this project leverages the low-latency, real-time processing capabilities that are essential for rhythm games. The ability to design custom hardware for input handling and timing synchronization ensures a highly responsive gameplay experience, a key feature that cannot be easily achieved with general-purpose processors.

\end{document}
