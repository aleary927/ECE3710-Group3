\documentclass{subfile} 

\begin{document}
  In order to communicate with the peripherals, the CPU needs some memory mapping so that it 
  can communicate using simple memory reads and writes.
  The memory-mapped I/O space in this design is very small, spanning less than 16 addresses. 
  These addresses are aligned at the top of the address space, so the most significant 
  12 bits of address space being high indicate that data should come from or be 
  written to an I/O register instead of memory.

  Nothing will happen if trying to write to a I/O address that is invalid, and if trying to 
  read from an invalid I/O address, the data will come back as all zeros.
  Writing to a read-only peripheral results in that data being lost, because it will be 
  written to memory but it will not be retrievable. 
  All peripherals which can be written to can also be read from as well.

  \begin{table}[h]
    \caption{Peripheral Addresses} 
    \label{tab:addrs}
    \centering
    \begin{tabular}{|c|c|}
      \hline 
      Peripheral & Address \\ \hline
      \hline
      Switches & 0xFFFF \\ \hline 
      Buttons & 0xFFFE \\ \hline 
      Leds    & 0xFFFD \\ \hline 
      Hex high & 0xFFFC \\ \hline 
      Hex low & 0xFFFB \\ \hline 
      VGA hCount & 0xFFFA \\ \hline 
      VGA vCount & 0xFFF9 \\ \hline 
      Music Control & 0xFFF8 \\ \hline 
      Drum Pads & 0xFFF7 \\ \hline 
    \end{tabular}
  \end{table}
  
\end{document}
