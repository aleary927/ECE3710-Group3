\documentclass{subfile} 

\begin{document}
The system achieved smooth synchronization with no evident latency and a flicker-free 
display of the game elements. 
Tiles moved at a consistent speed, in sync with the music, showcasing the precision timing and 
responsiveness of the FPGA-based design. 
The display of player scores and tile interactions behaved as expected. 
The trigger of drum pads associated with distinct sounds operated exceptionally well, with 
no noticeable delay or disruption; this, in conjunction with good quality 
audio streaming from the music, created an engaging auditory layer that was  
complemented by the visual elements of the game. 
These features demonstrated the system's ability to handle real-time input 
processing and graphical rendering concurrently. 

While the rhythm tile game demonstrated strong functionality in its final 
implementation and its core mechanics operated without issue, it was 
limited to a single level. 
The absence of additional levels posed no progressive difficulty for players. 
This limitation was primarily due to time constraints and a focus on 
ensuring the stability of the primary features, like VGA display synchronization, 
tile rendering, and drum pad coordination. The lack of extra levels 
could be remedied by expanding memory options external to the FPGA to allow for 
more audio samples to be accessed by the FPGA at once.
One way that this could be accomplished is by moving the audio samples to the HPS, 
averting the block ram resource limitation.

Another improvement that would have been nice to have is encoding exact note locations, rather 
than using contiguous identically-sized windows. 
This would have made the programming significantly more complicated however, requiring too 
much time.

Overall, the foundational framework of the game is robust and versatile and would be able to support the aforementioned additions. 

\end{document}
