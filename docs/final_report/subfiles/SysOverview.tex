\documentclass{subfile} 

\begin{document}
  This project involves several subsystems which all have to work together to create a 
  cohesive final system. 
  The main components are as follows: CPU, audio subsystem, VGA subsystem, memory subsystem, 
  and HPS.

  The CPU is the main controller, controlling the rest of the components via a memory-mapped interface. 
  The CPU is able to pause audio, reset audio, read inputs from the drum pads, write data to be read 
  by the VGA controller, and read and write from all the standard board peripherals (switches, leds, seven 
  segment displays, buttons).

  The HPS is used purely to stream a song to the FPGA. 
  Songs are quit a bit of data, so it is necessary to store the song external to the FPGA. 
  The HPS is a very nice solution because it allows for data to be read from a sound file hosted 
  on the sd-card, removing the need for extra external hardware.

  The role of the rest of the subsystems in the overall design are fairly self explanatory. 
  Each subsystem will receive an in-depth explanation in the following sections.
\end{document}
