\documentclass{subfile} 

\begin{document}
  This project involves several subsystems which all have to work together to create a 
  cohesive final system. 
  The main components are as follows: 
  \begin{itemize} 
  \item CPU 
  \item Audio subsystem 
  \item VGA subsystem 
  \item Memory subsystem, 
  \item HPS (hard processor system)
  \end{itemize}

  The CPU is the main controller, controlling the rest of the components via a memory-mapped interface. 
  The CPU has control of the audio, the ability to share data with the VGA controller, and access 
  to all board peripherals via memory-mapping.
  On the audio front, the CPU is able to pause audio, reset the soundtrack, and disable the 
  soundtrack as to only play sound effects.
  The board peripherals which the CPU can access include all on-board leds, buttons, and switches, 
  as well as the drum pad inputs.

  The HPS is used purely to stream a song to the FPGA. 
  A full song is a substantial amount of data, so it is necessary to store the song external to 
  the FPGA fabric. 
  The HPS is a very nice solution because it allows for data to be read from a sound file hosted 
  on the sd-card, removing the need for extra external hardware.
  The VGA and audio subsystems take care of outputting the proper signals to the VGA 
  controller and audio codec that are on-board the DE1-SoC board.
  The Memory subsystem contains memory shared between the CPU and VGA controller, and 
  also includes all memory-mapping.

\end{document}
